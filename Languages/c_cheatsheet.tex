\documentclass[10pt]{article}

%AMSLaTeX packages
\usepackage{amsthm}
\usepackage{amsmath}
\usepackage{amsfonts}

% we want to use images
\usepackage{graphicx}
\usepackage{alltt}
\usepackage{verbatim}
\usepackage{multicol}
\usepackage{tabularx}
\usepackage{a4wide}

\title{C Cheatsheet}
\author{Carl Ellis}

\begin{document}

\maketitle

\begin{abstract}
This document will cover the fundamentals of a C program, including stucture and commonly used library functions.
\end{abstract}

\section{Program Structure}
This section will describe the structure of a C program.

\begin{center}
    \begin{tabularx}{\columnwidth}{@{\extracolsep{\fill}} |l|X| }
      \hline
      Function prototype  & \texttt{type func($\text{arg}_1$, ...);} \\ \hline

      Function structure  & \texttt{type func($\text{arg}_1$, ...) \{ } \\
                          & ~ \textit{declarations} \\
                          & ~ \textit{statements} \\
                          & ~ \texttt{return value;} \\
                          & \} \\ \hline

      Main function       & \texttt{int main(int argc, char *argv[]) \{} \\
      (Program Entry)     & ~ \textit{declarations} \\
  (argv =  cmdline args)  & ~ \textit{statements} \\
  (argc = no. of args)    & ~ \texttt{return 0;} \\
                          & \} \\ \hline
      C file structure    & \textit{preprocessor declarations} \\
      (Program Entry)     & \textit{declarations} \\
                          & \textit{functions} \\ \hline
    \end{tabularx}
\end{center}

\section{Pre Processor Commands}
This section will describe the commands passed to the preprocessor.
These are processed at compile time, and describe static changes to the program. 
This is mostly syntactic sugar.

\begin{center}
    \begin{tabularx}{\columnwidth}{@{\extracolsep{\fill}} |l|X| }
      \hline
      Include a library file  & \texttt{\#include <}\textit{filename}\texttt{>} \\ \hline
      Include a user file     & \texttt{\#include \"}\textit{filename}\texttt{\"} \\ \hline
      Define a constant       & \texttt{\#define} \textit{name} \textit{value} \\ \hline
      Constant defined?       & \texttt{\#ifdef} \\
      not defined?            & \texttt{\#ifndef} \\ \hline
      Conditional execution   & \texttt{\#if}, \texttt{\#else}, \texttt{\#elif}, \texttt{\#endif} \\ \hline
    \end{tabularx}
\end{center}

\section{Data types}
This section will describe the data types supported by C. 
All variables must have a type associated with it, as this defines the memory allocation policy.

\begin{center}
    \begin{tabularx}{\columnwidth}{@{\extracolsep{\fill}} |l|X| }
      \hline
      No type                   & \texttt{void} \\ \hline
      Character (1 byte)        & \texttt{char} \\ \hline
      Short integer (2 bytes)   & \texttt{short} \\ \hline
      Integer (4 bytes)         & \texttt{int} \\ \hline
      Long integer (8 bytes)    & \texttt{long} \\ \hline
      Positive or negative      & \texttt{signed} \\ \hline
      Positive only             & \texttt{unsigned} \\ \hline
      Real number               & \texttt{float} \\
      Double Precision          & \texttt{double} \\ \hline
      Pointer to a type         & \textit{type}\texttt{*} \\ \hline
      Easy Structure            & \texttt{typdef struct \{} \\ 
      (Contains many types)     & ~ \texttt{type} \textit{name} \texttt{;} \\ 
                                & ~ \texttt{...} \\ 
                                & \texttt{ \}} \textit{name} \texttt{;} \\ \hline
    \end{tabularx}
\end{center}

\section{Operators}
Operators are syntactic sugar which allows simple functions to be called by used a few characters.
Operators can be thought of as functions which take arguments from either side of the function call.
So \texttt{x f y} can be thought of as \texttt{f(x, y)}.

\begin{center}
    \begin{tabularx}{\columnwidth}{@{\extracolsep{\fill}} |l|X| }
      \hline
      Comment                   & \texttt{/*}\textit{comment}\texttt{*/}, \texttt{//}\textit{comment} \\ \hline
      Assignment                & \texttt{=} \\ \hline
      Structure member access   & \texttt{struct.member} \\
                                & \texttt{structptr->member} \\ \hline
      Increment, decrement      & \texttt{++}, \texttt{--} \\ \hline
      Arithmetic operators      & \texttt{+} \\ 
\textit{Plus, minus, multiply}  & \texttt{-} \\ 
\textit{divide, modulus}        & \texttt{*} \\ 
                                & \texttt{/} \\ 
                                & \texttt{\%} \\ \hline
      Relational Comparisons    & \texttt{<} \\ 
                                & \texttt{<=} \\ 
                                & \texttt{>=} \\ 
                                & \texttt{>} \\ \hline
      Equality Comparisons      & \texttt{==} \\ 
                                & \texttt{!=} \\ \hline
      Logical Comparisons       & \texttt{\&\&} \\ 
\textit{And, or, not}           & \texttt{||} \\
                                & \texttt{!} \\ \hline
      Bitwise Comparisons       & \texttt{\&} \\ 
\textit{And, or, xor, not}      & \texttt{|} \\
                                & \texttt{\^} \\
                                & \texttt{\~} \\ \hline
      Bitwise Operators         & \texttt{<<} \\ 
      \textit{Bit shifts}       & \texttt{>>} \\ \hline
      Dereference pointer       & \texttt{*}\textit{ptr} \\ \hline
      Address of variable       & \texttt{\&}\textit{name} \\ \hline
      Size of variable          & \texttt{sizeof} \textit{name} \\ \hline
      Array access              & \texttt{array[}\textit{index}\texttt{]} \\ \hline
      Type cast                 & \texttt{(type)}\textit{name}\\ \hline
    \end{tabularx}
\end{center}

\section{Control Structures}
Control structures are what the programmer uses to bend reality unto their will.
Control flow is a matter of conditional and unconditional jumps.

\begin{center}
    \begin{tabularx}{\columnwidth}{@{\extracolsep{\fill}} |l|X| }
      \hline
      Call a function           & \texttt{functionname(}\textit{parameters}\texttt{);} \\ \hline
      Conditional- \textit{if}  & \texttt{if (}\textit{$\text{expression}_1$}\texttt{) \{} \\ 
                                & ~\textit{$\text{statement}_1$} \\ 
                                & \texttt{else if (}\textit{$\text{expression}_2$}\texttt{) \{} \\ 
                                & ~\textit{$\text{statement}_2$} \\ 
                                & \texttt{else \{} \\ 
                                & ~\textit{$\text{statement}_3$} \\ 
                                & \texttt{\}} \\ \hline
  Conditional- \textit{switch}  & \texttt{switch (}\textit{expression}\texttt{) \{} \\ 
                                & ~\texttt{case }\textit{$\text{const}_1$}\texttt{:}\textit{$\text{statement}_1$}\texttt{ break;} \\ 
                                & ~\texttt{case }\textit{$\text{const}_2$}\texttt{:}\textit{$\text{statement}_2$}\texttt{ break;} \\ 
                                & ~\texttt{default: }\textit{statement}\\ 
                                & \texttt{\}} \\ \hline
      Iterator - \textit{while} & \texttt{while (}\textit{expression}\texttt{) \{} \\ 
                                & ~\textit{statement} \\ 
                                & \texttt{\}} \\ \hline
      Iterator - \textit{do}    & \texttt{do \{} \\ 
                                & ~\textit{statement} \\ 
                                & \texttt{\} while (}\textit{expression}\texttt{);} \\ \hline 
      Iterator - \textit{for}   & \texttt{for (}\textit{init}\texttt{;}\texttt{condition}\texttt{;}\textit{incr}\texttt{) \{} \\ 
                                & ~\textit{statement} \\ 
                                & \texttt{\}} \\ \hline
    \end{tabularx}
\end{center}

\section{Common functions}
This are functions which you will use a lot, normally part of one of the system libraries.

\begin{center}
    \begin{tabularx}{\columnwidth}{@{\extracolsep{\fill}} |l|X| }
      \hline
      Allocate memory on the heap   & \texttt{void * malloc(int size);} \\ \hline
      Free memory                   & \texttt{void free(void * name} \\ \hline
      Print to STDOUT               & \texttt{printf(}\textit{format\_string}\texttt{, $\text{arg}_1$, ...);} \\ 
      Format string uses key        & \%d - int, \%c - character \\ 
      substitution to print         & \%s - string \\ 
      variables by type.            & \textit{(see example code)} \\ \hline
    \end{tabularx}
\end{center}

\section{Example}
Here is an example program, which calls a function dependent upon the command line arguments given.

\begin{center}
    \begin{tabularx}{\columnwidth}{@{\extracolsep{\fill}} |X| }
      \hline
\begin{alltt}
\#include<stdlib.h> 

\#include<stdio.h>


/* Function Prototypes */

void print\_number(int number, char * arg);


/* Functions */

void print\_number(int number, char * arg)\{

~  printf("The arg number is \%d, the arg is \%s \textbackslash n", number);

\}


/* Main function */

int main(int argc, char * argv$[$ $]$) \{

  
~  /* Variables */

~  int i;


~  /* If not command line arguments given then exit */

~  if ( argc == 1 ) \{

~  ~ return 0;

~  \}

  
~  /* loop the amount of arguments */

~  for(i=0;i<argc;i++) \{

~  ~  print\_number(i,argv$[$i$]$);

~  \}


~  return 0;

\}

\end{alltt}
    \\  \hline
    \end{tabularx}
\end{center}

\end{document}

